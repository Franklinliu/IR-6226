\section{Evaluation}
In this section, our information retrieval system is to evaluate on the measure time for creating an index and answering shorter and longer query respectively. And also the memory usage is to calculate for holding the index. We will compare the index size with document corpus size as well. Finally, we will demonstrate the correctness of information retrieval system on a small documents set.
\paragraph{Refine the dataset.} Original Dataset contains much irrelevant information such as email comment which indicates email is unclassified and  associated with U.S. Department of State. To better present information related to email body, we removed those message and store 
emails into XML documents amenable to quickly index the email. Table~\ref{tab:refine} shows that out of 7945 emails, we clarified 4816 emails while the others was discarded for some format problems such as incompatible to store as XML document.
\begin{table}[!ht]
	\centering
	\begin{tabular}{c|ccc}
		\toprule
		 & orginal & error & selected \\
		\midrule
		 number & 7945  & 3129 & 4816	\\
		\bottomrule
	\end{tabular}
	\caption{Refine the HillaryEmails dataset}
	\label{tab:refine}
\end{table}
%\todo{liuye}
\paragraph{Measure the time.}  As shows in Table~\ref{tab:time}, we evaluated how much time was used for creating for each index and for varied sized query(len=3,5,7,10). It is clear that creating an index is fast and accounts for 1.46 millisecond. When the length of query varies, the time used varies as well. Answering query with length of 3 spent 0.73 milisecond at the lowest while that of 10 spent 3.93 millisecond at the highest. And this demonstrates that answering less length of query spent less time. Overall, both time used for creating an index and answer an query is relatively fast.
\begin{table}[!ht]
	\centering
	\begin{tabular}{c|c|cccc}
		\toprule
		&\multirow{2}{*}{index} &  \multicolumn{4}{c}{query} \\
		\cline{3-6}
		         &    &len=3 & len=5 & len=7 & len=10 \\
		\midrule
	    time(ms)&  1.46 & 0.73  & 1.92  & 2.52  & 3.93	\\
		\bottomrule
	\end{tabular}
	\caption{Time for creating index and answering query}
	\label{tab:time}
\end{table}

\paragraph{Measure the size.} Table~\ref{tab:size} shows that size of HillaryEmails dataset. Size of original HillaryEmails dataset is 40MB. After selection, we focused on part of the dataset taking up 37MB. We created 42364 indices for this dataset. For storing the indices in text format, we have 1.5MB storage to maintain it. However, storing the indices using trie (patricia trie) reduced the storage hugely to 218KB. Thus, choosing different structure or format to store index can make a difference with respect to the memory or storage requirement accordingly. 
\begin{table}[!ht]
	\centering
	\begin{tabular}{c|ccccc}
		\toprule
		        & orginal & error & selected & index(text) & index(trie)\\
		\midrule
		number  & 7945    & 3129  & 4816     & 42364   & 42364\\
		size    & 40MB   & 13M   & 37MB      & 1.5MB   & 218KB \\
		\bottomrule
	\end{tabular}
	\caption{Size of the index and the HillaryEmails dataset}
	\label{tab:size}
\end{table}

\paragraph{Verify the result.} To illustrate how much of correctness the information retrieval system could achieve, as shows in Table~\ref{tab:email}, we selected an email from the dataset.
We construct several query string to the information retrival system to verify whether the results are correct or not in Table~\ref{tab:query}. \todo{liuye}
\begin{table}[!ht]
	\centering
	\begin{tabular}{r|c}
		\toprule
		q\_id& 1 \\
		\midrule
		query & justice sensitive Muslims\\
		terms &  \\ 
		relevant docs & \\
		correctness & \\
		\bottomrule
	\end{tabular}
	\caption{Verify the query result}
	\label{tab:query}
\end{table}
\begin{table*}[t]
	\begin{tabular*}{\textwidth}{c|c}
		\toprule
		doc&\texttt{HillaryEmails-5898.txt} \\
		\midrule
		header&\multirow{8}{.9\textwidth}{\texttt{
		From: Abedin, Huma <AbedinH@state.goy>
		Sent: Thursday, August 19, 2010 12:42 PM
		To:
		Subject: Fw: Indian PM shows solidarity with Pakistan against floods
		From: Sullivan, Jacob 3
		To: Abedin, Huma
		Sent: Thu Aug 19 11:07:01 2010
		Subject: FW: Indian PM shows solidarity with Pakistan against floods
		Can you flag for S —Singh is doing a good thing here.
		From: membership\_services@fma.sosiltd.com [mailto:membership\_services@fma.sosiltd.com] On Behalf Of News Desk
		Sent: Thursday, August 19, 2010 10:55 AM
		To: Afghanistan-Pakistan News Alerts; CENTCOM News Alerts; PACOM News Alerts
		Subject: Indian PM shows solidarity with Pakistan against floods
	}} \\
	&    \\
	&    \\
	&    \\
	&    \\
	&    \\
	&    \\
	&    \\
	&    \\
     \midrule 
        body&\multirow{20}{.9\textwidth}{\texttt{
		Indian PM shows solidarity with Pakistan against floods
		ISLAMABAD, Aug. 19 (Xinhua) -- Indian Prime Minister Manmohan Singh called his Pakistani
		counterpart Syed Yusuf Raza Gilani over phone Thursday to convey the condolences and
		commensurations of his government and the people of India on the loss of precious lives
		and damage to the properties and infrastructure caused by floods in Pakistan.
		The Indian prime minister said that the member States in the South Asian Association for
		Regional Cooperation should come together to help Pakistan in every way possible at this
		critical juncture, according to the Pakistani PM office.
		Gilani thanked his Indian counterpart for his kind gesture of calling him to express his
		sympathies and solidarity with Pakistan in its hour of trial.
		He briefed the Indian prime minister over the huge damage caused by this unprecedented
		natural calamity in all the provinces of the country, which, according to UN estimates,
		was far bigger than the effects of earthquakes and tsunami which hit Pakistan in 2005 and
		Haiti last year respectively.
		Gilani also availed the opportunity to reiterate Pakistan's keen desire to have friendly,
		cooperative and good neighborly relations with India.
		He hoped that in accordance with the agreement reached in Thimphu in Bhutan in April, the
		bilateral dialogue between the two countries would progress a meaningful, substantive and
		result- oriented way by addressing all the issues of concern to both countries including
		Kashmir dispute.
		Media Analysis and Watch Center
		USSTRATCOM Foreign Media Analysis Program
		SOS International Ltd.
         }} \\ 
        &    \\
        &    \\
        &    \\
        &    \\
        &    \\
        &    \\
        &    \\
        &    \\
        &    \\
        &    \\
        &    \\
        &    \\
        &    \\
        &    \\
        &    \\
        &    \\
        &    \\
        &    \\
        &    \\
        &    \\
	    \bottomrule
	\end{tabular*}
	\label{tab:email}
\end{table*}


\section{Optimization}
\todo{liuye}